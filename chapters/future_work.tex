\chapter{Future Directions}

We list here a few directions in which this work could be continued:

\begin{itemize}
    \item We identified in note \ref{note:delta_meaning} that we always assumed $\Delta$
        referred to the maximum degree function. There is no reason to believe that this
        is the only graph parameter which could be used. For example, consider the maximum codegree
        function $\Delta_2$ (the max size of $N(u)\cap N(v)$ for any $u,v\in V(G)$).
        Then the flag $\codegreeflag$ has the property that
        $c(\codegreeflag; (G,\eta)) \leq \Delta_2(G)$
        for any embedding $\eta$. Hence choosing this function as our normalisation function
        makes $\codegreeflag$ a local flag.
    \item In this thesis we focused entirely on flags as applied to simple graphs. However,
        flags have been applied more broadly, including both closely related concepts such
        as directed graphs \cite{gilboaLocalStructureOriented2022},
        permutations \cite{baloghMinimumNumberMonotone2015} and discrete geometry
        \cite{goaocLimitsOrderTypes2018}. An interesting line of investigation would be
        to see if this method can be adapted to those areas.
    \item In chapter \ref{chap:pentagon_conjecture} we investigated bounding pentagons
        in a triangle free graph. This method could be adapted to other problems in this
        area, referred to as \textit{generalised Turán numbers}. e.g. Bounding how many
        $C_7$s you can have in a $C_5$ free graph.
    \item In chapter \ref{chap:pentagon_conjecture} we showed that $P(G, v) \lesssim 1/8$
        for $G$ triangle free, and showed that this bound is tight (lemma
        \ref{lemma:pentagon_1_8_tight}). A natural question to ask is whether this
        extremal graph is unique.
    \item We defined a local type (definition \ref{def:local_type}) so that we had
        a well defined, positivity preserving, averaging operator (lemma
        \ref{lemma:local_pos_preserve}). However we can note that the averaging lemma
        (lemma \ref{lemma:local_averaging_exp}) does not depend on $\sigma$ being a
        local type. There is a sense in which we can still "unlabel" these flags.
        For example, if $\Gcl$ is all graphs then $\cfivemarked$ is a local $\vertex$-flag
        but $\cfive$ is not a local $\emptyset$-flag. However, if we introduce a new
        "density function" on $\emptyset$-flags as 
        \[
            \zeta(F; G) := \frac{c(F; G)}{|G|\binom{\Delta(G)}{|F|-1}}
        \]
        Then we believe we can linearly extend this over $\R\Gcl$ and define limit functionals.
        Then applying lemma \ref{lemma:local_averaging_exp} similar to the proof of
        lemma \ref{lemma:local_pos_preserve} we can show that $f \geq 0$ does imply
        $\zeta(\llbracket f \rrbracket; G) \gtrsim 0$. We believe this approach could
        improve even further on the bound on the pentagon conjecture.
\end{itemize}
