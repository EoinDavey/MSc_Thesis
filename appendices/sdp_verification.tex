\chapter{SDP Verification}

In section \ref{sec:pentagon_proof_validation} we showed the results converting an
SDP solution back into a standalone proof. This was mostly for elucidatory purposes,
confirming that the SDP solutions do indeed correspond to a convex combination of elements
of the semantic cone.
I will give a brief overview of how I did this conversion here.
However, the process of doing this conversion is tedious and often
unnecessary. I will also describe how we can avoid doing this conversion for other problems.

\section*{Conversion to Standalone Proof}

The SDP software outlined in appendix \ref{app:flag_software} contains some
functionality to find a minimised set of constraints. It does this by
running the search with all constraints, then removing them one at a time so long as
the optimum solution is maintained. The function which does this is
\verb|FlagSolver::minimize|. There are several such methods implemented but I
only verified the correctness of \verb|minimize| as implemented in the
\url{github.com/EoinDavey/rust-flag-algebra} fork of the \verb|rust-flag-algebra|
library.

\begin{note*}
    CSDP will fail to run if there are any variables which have no associated constraints.
    For that reason I recommend using the \verb|.protect| method to prevent the minimising
    process from removing the $F \geq 0\ \forall F$ inequalities.
\end{note*}

This minimisation process reduces the size of the dual matrices that we need to deal with.
However, if the search process itself is slow, then this process will be extremely slow.
Hence it has limited applications for larger problems.

Given then a dual certificate (minimised or not) from our SDP program it will be in
SDPA-Sparse format. This is a sparse block-diagonal format of describing a matrix.
These blocks then correspond to the coefficients associated with our inequalities. i.e.
they are blocks of size 1 for linear inequalities or PSD matrices for the Cauchy-Schwarz
$\llbracket f^2 \rrbracket \geq 0$ inequalities. In the
\url{github.com/EoinDavey/rust-flag-algebra} repository you will find a
Python\footnote{\url{https://python.org/}} script \verb|sdpa_parse.py| which can
interpret the SDPA problem input file and certificate file.

At this point the main goal is to remove floating point inaccuracy. In practice you
will find that most coefficients are clearly floating point approximations of identifiable
rationals, $1/4$, $7/5$ etc. To do this process of replacing floating point approximations with
their associated rational I copied the \verb|sdpa_parse.py| file and modified it to
overwrite the associated coefficients, while asserting that we still get an
objective function bound close to what we expected.
This was a tedious manual process but after some time I had converted the coefficients to
nice rationals.

This process includes replacing the entries of the PSD matrices with rationals. This meant
I needed to check that each such matrix was still PSD. After obtaining these rational
PSD matrices I wanted to extract the associated vectors $\nu_i$ which give us the
elements $\llbracket \nu_i^2 \rrbracket$ as described in section \ref{sec:sdp_duality}.
The most direct way to do this is to compute the spectral decomposition, I used
\url{https://wolframalpha.com/} to do this symbolically. However, one of the PSD matrices
gave me some "ugly" coefficients such as $\sqrt{11 + \sqrt{89}}$. For that matrix
I instead used the Z3 Solver\footnote{\url{https://www.microsoft.com/en-us/research/project/z3-3/}}
to find a nicer set of vectors $\nu_i$ such that $\sum_i \nu_i\nu_i^T$ gave the same
matrix. This was just an aesthetic choice.

\begin{note*}
    When researching this project I found the paper
    \textit{Exact Semidefinite Programming Bounds for Packing Problems} by
    Dostert, de Laat and Moustrou \cite{dostertExactSemidefiniteProgramming2021}
    which describes a more mathematical approach to making a floating point SDP solution
    exact. I didn't use this approach but I think it's valuable to mention it here.
\end{note*}

\section*{Verification Without Conversion}

Most of the results in this thesis are not "tight", in the sense that we are rounding
the SDP solutions up. e.g. In lemma \ref{lemma:sec_sdp_soln} we give the result that
$10.644\emptyset - O \in\SemCone^\emptyset$, but in fact the SDP solver claims a
better result of $10.643189$. Rounding up gives us some room to maneuver with respect
to the floating point, while still achieving the nice $1.73$ bound of theorem
\ref{thm:strong_edge_colouring_bound}. For this reason we need only to take the
dual solution, round to some level of precision and verify that this is still a PSD matrix
using full precision arithmetic.
Then we can calculate the associated bound and verify that it still is low enough, along
with verifying the inequalities still hold.
