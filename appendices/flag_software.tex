\chapter{Software Implementation}
\label{app:flag_software}

Our implementations of the SDP programs are built on the flag software written by
Rémi de Verclos. In particular there are the following two projects:
\begin{itemize}
    \item An implementation of the classic flag algebras based on work done during his
        PhD (\cite{deverclosApplicationsLimitsCombinatorial2016})
        \url{https://github.com/avangogo/rust-flag-algebra}.
    \item A library using the previous implementation written to test if the local flags idea
        would work in principle: \url{https://github.com/avangogo/local-flags}.
\end{itemize}

The modifications made to these projects for this thesis can be found in the following forks:
\url{https://github.com/EoinDavey/rust-flag-algebra} and\\
\url{https://github.com/EoinDavey/local-flags}.
These implementations use Rust\footnote{\url{https://www.rust-lang.org/}} to generate SDP programs which are
then solved using CSDP\footnote{\url{https://github.com/coin-or/Csdp/wiki}},
a standard open source SDP solver
\cite{borchersCSDPLibrarySemidefinite1999}.

\begin{note}
    Rust is a good choice of language for this application. It is a strongly, statically typed
    language with high level constructions meaning we can define concepts such as flags, types
    etc directly in the code. It is however also extremely fast, meaning the bulk computation
    of matrices of coefficients over large bases is still extremely fast.
\end{note}

The implementations for the applications in this thesis can be found at these locations:
\begin{itemize}
    \item The "direct" pentagon bounding result, lemma
        \ref{lemma:pentagon_1_4_bound}:\\
        \verb|local-flags/examples/bounded_pentagon.rs|.
    \item The stronger pentagon bounding result, lemma
        \ref{lemma:pentagon_stronger_bound}:\\
        \verb|local-flags/examples/bounded_pentagon_alt_approach.rs|.
    \item The strong edge colouring conjecture bound, lemma
        \ref{lemma:sec_sdp_soln}:\\
        \verb|local-flags/examples/strong_density.rs|.
    \item The bipartite strong edge colouring conjecture bound, lemma
        \ref{lemma:sec_bipartite_sdp_soln}:\\
        \verb|local-flags/examples/bipartite_strong_density.rs|.
\end{itemize}

\noindent At an extremely high level, the local flag algebra programs do the following steps:
\begin{enumerate}
    \item Define the class of local $\emptyset$-flags by defining a sub-class of
        coloured graphs.
    \item Compute an ordered basis of flags of fixed size $n$.
    \item Construct the objective vector over this basis.
    \item Enumerate linear constraint vectors over this basis.
    \item Calculate all local types $\sigma$ and sizes $k$ such that $f^2$ is a vector
        of size $n$ if $f \in \Lcl^\sigma_m$. Generate the coefficient matrices
        $C^\sigma_i$ as described in section \ref{sec:semidefinite_method}.
    \item Write all these vectors and matrices in standard format (SDPA Sparse) such that it
        can be used with CSDP.
    \item Call CSDP as a subroutine to solve the generated SDP problem.
    \item (Optionally) read the generated SDP solution and generate a HTML report.
\end{enumerate}

After these programs have completed and found an optimal bound for the SDP
they will create a \textit{certificate} file containing the optimal (dual and primal)
solution in SDPA-Sparse format. This certificate contains the linear coefficients
and PSD matrices needed to extract a rigorous proof that $\lambda\emptyset - O\in\SemCone^\emptyset$
where $\lambda$ is our optimal bound and $O$ is the objective vector. We give more
information on recovering the proof in appendix TODO.

\section*{Example}

We show now some of the details of how the proof of lemma \ref{lemma:pentagon_1_4_bound}
is implemented. I'll skip all boilerplate and show only the code which is of high
importance:

First we have some code describing the local $\emptyset$-flags which are those
where each connected component contains a black vertex and there are no
triangles or edges between black vertices.

\begin{lstlisting}[basicstyle=\scriptsize]
impl SubFlag<G> for TriangleFreeConnected {
    const HEREDITARY: bool = false;

    fn is_in_subclass(flag: &G) -> bool {
        // Each connected component contains a vertex colored 0
        if !flag.is_connected_to(|i| flag.color[i] == 0) {
            return false;
        }
        // No black-black edges.
        if flag.content.edges()
            .any(|(u, v)| flag.color[u] == 0 && flag.color[v] == 0) {
            return false;
        }
        // No triangles.
        let n = flag.content.size();
        for (u, v, w) in iproduct!(0..n, 0..n, 0..n) {
            if u == v || u == w || v == w {
                continue;
            }
            if !flag.content.edge(u, v)
                || !flag.content.edge(u, w)
                || !flag.content.edge(v, w) {
                continue;
            }
            return false;
        }
        true
    }
}
\end{lstlisting}

We then create the basis of flags of size 5 and construct our
objective vector $O=\llbracket \brrb \cdot \ext_1^\sigma \rrbracket$.
We use a helper function \verb|Degree::project| to construct the extension vectors.

\begin{lstlisting}[basicstyle=\scriptsize]
pub fn main() {
    let n = 5;
    let basis = Basis::new(n);

    let c5_path: F = Colored::new(
            Graph::new(4, &[(0, 1), (1, 2), (2, 3)]), vec![0, 1, 1, 0])
            .into();
    let obj = Degree::project(&c5_path, n).untype();
\end{lstlisting}

Then we create a vector of inequalities, which we fill with the inequalities
meaning $\phi(F) \geq 0$ for all flags, $\phi(\llbracket B_k \cdot \ext^{B_k}_1 \rrbracket) = 1$
and the regularity constraints
$\phi(\llbracket \ext_i^\sigma - \ext_j^\sigma\rrbracket) = 0$.
\begin{lstlisting}[basicstyle=\scriptsize]
let mut ineqs = vec![flags_are_nonnegative(basis)];
for i in 1..=n {
    ineqs.push(ones(n, i).untype().equal(1.));
}

ineqs.append(&mut Degree::regularity(basis));
\end{lstlisting}

Finally we construct a \verb|Problem| object with these linear constraints and
all $\phi(\llbracket f^2\rrbracket) \geq 0$ constraints\footnote{These are referred to
as Cauchy-Schwarz inequalities in the codebase}.
We then use the convenience object \verb|FlagSolver| to execute this problem
and to print a HTML report.

\begin{lstlisting}[basicstyle=\scriptsize]
let pb = Problem::<N, _> {
    ineqs,
    cs: basis.all_cs(),
    obj: -obj,
}
.no_scale(); // Means we don't have to rescale the output.

let mut f = FlagSolver::new(pb, "bounded_pentagon");
f.init();
f.print_report(); // Write some informations in report.html
\end{lstlisting}

Executing this code with \verb|cargo run --release --example bounded_pentagon|\\
prints output \verb|Optimal value: 0.25000001| which is $1/4$ up to floating point error
as expected. The program writes the dual and primal solutions to disk as a file
\verb|certificate|, and the HTML report as \verb|report.html|.

The report should allow a quick visual check that the flag vectors look as expected.
This is most useful for small $n$, as for large $n$ the flags vectors are too large.

\begin{note}
    The programs generate a lot of precomputed data which is stored locally in a
    directory \verb|data/|. This can be quite a large amount of data, so users should
    take care to not run flag programs for $n$ too large unchecked, as they could in
    theory use up all disk space.
\end{note}

\section*{Notes on Correctness}

There are a few features and properties of the flag-algebra library which are valid
to use for classic flag algebras, but either are not valid for local flags or
we did not prove that they were valid.
\begin{itemize}
    \item Some operations in the flag-algebra library assume that the sum of flags
        of a fixed size is always equal to 1, which is not true for local flags.
        Having\\
        \verb|HEREDITARY = false| in the graph class definition should disable
        these operations.
    \item The \verb|rust-flag-algebra| library as written by de Verclos has some behaviour where it
        modifies some of the Cauchy-Schwarz matrices. I could not identify why it does
        this but I think it was intended to reduce the size of the generated SDP
        programs by taking advantage of some invariants across flags. As I could not validate what
        this behaviour was doing I disabled it in my fork of this library. Hence it is important
        (unless you can identify what these modifications are doing) to use the modified
        version of the \verb|rust-flag-algebra| library.
    \item There is a function \verb|multiply_and_unlabel| in the \verb|rust-flag-algebra|
        library which I did not prove was correct for local flags, but I suspect it is.
\end{itemize}
